\documentclass[12pt, a4paper]{article}
\usepackage[utf8]{vietnam}
\usepackage[left=2cm, right=3cm, bottom=2cm, top=2cm]{geometry}
\usepackage[table]{xcolor}

\definecolor{lightgreen}{RGB}{191, 255, 128}
\definecolor{lightergreen}{RGB}{230, 255, 204}

% working with tables
\usepackage{array}

\begin{document}

%Bảng luyện tập
%\begin{table}[h!]
%\begin{center}
%    \renewcommand{\arraystretch}{4}
%    \begin{tabular}{||m{5cm}|m{5cm}|m{5cm}||}
%    \textbf{Nhân tố sinh thái} & \textbf{Nhóm thực vật} & \textbf{Nhóm động vật} \\ \hline \hline
%    \hline
%    \textbf{Ánh sáng} & - Nhóm cây ưa sáng & Nhóm động vật tán sáng \\
%    \hline
%    \textbf{Nhiệt độ} & Thực vật biến nhiệt & Động vật biến nhiệt \\
%    \hline
%    \textbf{Độ ẩm} & Thực vật & Động vật \\
%    \hline
%    \end{tabular}
%    \caption{Phân chia các nhóm sinh vật}
%    \label{tab:my_label}
%\end{center}
%\end{table}

\begin{table}[h]
    \renewcommand{\arraystretch}{3}
    \begin{tabular}{||m{5cm}|m{5cm}|m{5cm}||}
    \hline
    \textbf{Nhân tố sinh thái} & \textbf{Nhóm thực vật} & \textbf{Nhóm động vật} \\
    \hline \hline
    \textbf{Ánh sáng} &
    - Nhóm cây ưa sáng \newline - Nhóm cây ưa bóng &
    - Nhóm động vật ưa sáng \newline - Nhóm động vật ưa tối \\
    \hline
    \textbf{Nhiệt độ} &
    - Thực vật biến nhiệt \newline - Thực vật hằng nhiệt &
    - Động vật biến nhiệt \newline - Động vật hằng nhiệt \\
    \hline
    \textbf{Độ ẩm} &
    - Thực vật ưa ẩm \newline - Thực vật chịu hạn &
    - Động vật ưa ẩm \newline - Động vật ưa khô \\
    \hline
    \end{tabular}
    \caption{Phân chia các nhóm sinh vật}
\end{table}
Bảng 1 cho ta thấy sự phân chia các nhóm sinh vật dựa vào các nhân tố sinh thái.

\newpage
{
    \arrayrulecolor[HTML]{00cc00}
    \setlength{\arrayrulewidth}{0.5mm}
    \setlength{\tabcolsep}{18pt}
    \renewcommand{\arraystretch}{2.5}
    \rowcolors{3}{lightgreen}{lightergreen}
    \begin{tabular}{|m{3cm}|m{3cm}|m{3cm}|}
        \hline
        \rowcolor[HTML]{99ff80}\multicolumn{3}{|c|}{Country List} \\
        \hline
        Country Name or Area Name& ISO ALPHA 2 Code &ISO ALPHA 3 \\
        \hline
        Afghanistan    & AF & AFG \\
        Aland Islands  & AX & ALA \\
        Albania        & AL & ALB \\
        Algeria        & DZ & DZA \\
        American Samoa & AS & ASM \\
        Andorra        & AD & AND \\
        Angola         & AO & AGO \\
        \hline
    \end{tabular}
}
\end{document}