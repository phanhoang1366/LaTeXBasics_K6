\documentclass[a4paper, 12pt]{report}
\usepackage[utf8]{vietnam}

\renewcommand{\familydefault}{\ttdefault}
\usepackage[hidelinks, colorlinks=true, linkcolor=red]{hyperref}
\usepackage[left=2cm, right=3cm, top=3cm, bottom=3cm]{geometry}

\usepackage{xcolor}
\usepackage{titlesec}

\usepackage{indentfirst}
\setlength{\parindent}{2em}

\usepackage{draftwatermark}
\SetWatermarkText{HARRY POTTER}
\SetWatermarkScale{4}
\SetWatermarkFontSize{20pt}
\SetWatermarkColor{red!20}

\usepackage{fancyhdr}
\pagestyle{fancy}
\fancyhf{}

\fancyhead[L]{\textbf{LCĐ - LCH Viện Toán ứng dụng và Tin học}}
\fancyhead[R]{\textbf{Harry Potter}}
\fancyfoot[C]{\thepage}


\renewcommand{\headrulewidth}{2pt}
\renewcommand{\footrulewidth}{1pt}

\begin{document}
\setcounter{page}{0}

\tableofcontents

\newpage
% make this page #f5f5d6
\pagecolor[HTML]{f5f5d6}

\chapter{Đứa bé vẫn sống}
\section{Đoạn văn 1}
{\fontsize{20pt}{25pt}
\selectfont
Ông bà Dursley, nhà số 4 đường Privet Drive, tự hào mà nói họ hoàn toàn bình thường, cám ơn bà con quan tâm. Bà con đừng trông mong gì họ tin vào những chuyện kỳ lạ hay bí ẩn, đơn giản là vì họ chẳng hơi đâu bận tâm đến mấy trò vớ vẩn đó.}\\

Ông Dursley là giám đốc một công ty gọi là Grunnings, chuyên sản suất máy khoan. Ông là một người cao lớn lực lưỡng, cổ gần như không có, nhưng lại có một bộ ria mép vĩ đại. Bà Dursley thì ốm nhom, tóc vàng, với một cái cổ dài gấp đôi bình thường, rất tiện cho bà nhóng qua hàng rào để dòm ngó nhà hàng xóm. Hai ông bà Dursley có một cậu quý tử tên là Dudley, mà theo ý họ thì không thể có đứa bé nào trên đời này ngoan hơn được nữa.

\textcolor{red}{
\section{Đoạn văn 2}
Gia đình Dursley có mọi thứ mà họ muốn, nhưng họ cũng có một bí mật, và
nỗi sợ hãi lớn nhất của họ là cái bí mật đó bị ai đó bật mí. Họ sợ mình sẽ
khó mà chịu đựng nổi nếu câu chuyện về gia đình Potter bị người ta khám phá.
Bà Potter là em gái của bà Dursley, nhưng nhiều năm rồi họ chẳng hề gặp gỡ
nhau. Bà Dursley lại còn giả đò như mình không có chị em nào hết, bởi vì cô
em cùng ông chồng vô tích sự của cô ta chẳng thể nào có được phong cách của
gia đình Dursley.
}

\section*{Đoạn văn 3}
\addcontentsline{toc}{section}{Đoạn văn 3}
Ông bà Dursley vẫn rùng mình ớn lạnh mỗi khi nghĩ đến chuyện hàng xóm sẽ
nói gì nếu thấy gia đình Potter xuất hiện trước cửa nhà mình. Họ biết gia đình
Potter có một đứa con trai nhỏ, nhưng họ cũng chưa từng nhìn thấy nó. Đứa bé
đó cũng là một lý do khiến họ tránh xa gia đình Potter: Họ không muốn cậu quý
tử Dudley chung chạ với một thằng con nít nhà Potter.

\newpage
% #e6e6ff
\pagecolor[HTML]{e6e6ff}

\chapter*{Chương 2\\Tấm kính biến mất}
\addcontentsline{toc}{chapter}{\protect\numberline{2}Chương 2}

\newpage
\

\newpage
% #fff2f2
\pagecolor[HTML]{fff2f2}
\chapter*{Chương 3\\Những lá thư không xuất xứ}
\addcontentsline{toc}{chapter}{\protect\numberline{3}Chương 3}

\newpage
\

\newpage   
% #ffffff
\pagecolor[HTML]{ffffff}
\chapter*{Chương 4\\Người giữ khóa}
\addcontentsline{toc}{chapter}{\protect\numberline{4}Chương 4}

\newpage
\

\newpage
\setcounter{chapter}{4}
\chapter{Hẻm Xéo}

\newpage
\

\newpage
\chapter{Hành trình từ sân ga chín - ba - phần - tư}

\newpage
\

\newpage
\chapter{Chiếc nón phân loại}

\end{document}