\documentclass[a4paper,12pt]{article}
\usepackage[utf8]{vietnam}
\usepackage[left=2cm, right=3cm, bottom=2cm, top=2cm]{geometry}
\usepackage{graphicx}
\usepackage{amsfonts, amssymb, amsthm}
\usepackage{mathtools}
\usepackage{xcolor}
\usepackage{hyperref}
\usepackage{algorithm, algorithmicx}
\usepackage[noend]{algpseudocode}
\usepackage{listings}

\newcommand{\varx}{\mathbf{x}}
\newcommand{\boldsigma}{\boldsymbol{\Sigma}}
\newcommand{\varmu}{\boldsymbol{\mu}}

\begin{document}
\textbf{Bài 1.} 

\begin{equation*}
    x^2 + y^2 + z^2 = 0 \tag*{I}
\end{equation*}

\textbf{Bài 2.} 

\begin{multline*}
    F = \{F_x \in F_c : (|S| > |C|) \cap \\
     (M_{inPixels} < |S| < max_{inPixels}) \cap \\
      (|S_{connected}| > |S| - \epsilon)\} \tag*{(1)}
\end{multline*}

\textbf{Bài 3.} 
\begin{align*}
    x &= 1 & xy &= 2 & xyz &= 3 \\
    xy &= 23 & xyz &= 3 & x &= 11 \\
    xyz &= 345 & x &= 1 & xyz &= 22
\end{align*}

\textbf{Bài 4.} 
\begin{align*}
    \Aboxed{x_1x_2x_3 = \frac{-d}{a}} & \tag*{[2]} \\
    x_1x_2 + x_2x_3 + x_3x_1 = \frac{c}{a} & \tag*{[3]}
\end{align*}

Đồng thời:

\begin{equation*}
    x_1 + x_2 + x_3 = \frac{-b}{a} \tag*{[4]}
\end{equation*}

\textbf{Bài 5.} 
\begin{equation*}
    \overbrace{\overset{+}{A}+\underset{-}{B} + {{A^B}^C}^2 + \cfrac{\omega}{\Phi}}^{\text{Vế trái}} = \underbrace{\underset{+}{A} + \underset{*}{B} + 12\cfrac{A^B}{C} + \cfrac{\Omega}{\varphi }}_{\text{Vế phải}} \tag*{(Pt.5)}
\end{equation*}

\newpage

\textbf{Bài 6.}
\begin{equation*}
    \frac{(\sqrt{3} + \sqrt{2})}{\sqrt{5} - \sqrt{7}} = 
    \frac{
        \begin{split}
            (\sqrt{3} + \sqrt{2})(\sqrt{5} + \sqrt{7}) \\
            (\sqrt{5} + \sqrt{7})
        \end{split}
    }{
        \begin{split}
            (\sqrt{5} - \sqrt{7})(\sqrt{5} + \sqrt{7}) \\
            (\sqrt{5} + \sqrt{7})
        \end{split}
    } \tag*{(2)}
\end{equation*}

\lstdefinestyle{my_style}{
    language=Java,
    backgroundcolor=\color{green!10!},
    numberstyle=\tiny\color{gray},
    basicstyle=\ttfamily\small,
    keywordstyle=\color{blue},
    commentstyle=\color{green},
    numberstyle=\tiny\color{gray},
    stringstyle=\color{red},
    numbers=left,
    numbersep=5pt,
    frame=single,
    breaklines=true,
    breakatwhitespace=true,
    showstringspaces=false,
    tabsize=10,
    captionpos=t
}

\lstset{style=my_style}

\textbf{Bài 7.}
\begin{lstlisting}[language=Python, caption={Mã nguồn Python}, label={lst:python-code}]
def factorial(n):
    if n == 0 or n == 1:
        return 1
    else:
        return n * factorial(n - 1)

num = 5
result = factorial(num)
print(f"The factorial of {num} is {result}.")
\end{lstlisting}

\textbf{Bài 7.} \\
\begin{algorithm}[H]
    \caption{QuickSort}
    \begin{algorithmic}[1]
        \Procedure{QuickSort}{$arr, low, high$}
            \If{$low < high$}
                \State $pivot \gets \Call{Partition}{arr, low, high}$
                \State $\Call{QuickSort}{arr, low, pivot - 1}$
                \State $\Call{QuickSort}{arr, pivot + 1, high}$
            \EndIf
        \EndProcedure
        \newline
        \Procedure{Partition}{$arr, low, high$}
            \State $pivot \gets arr[high]$
            \State $i \gets low - 1$
            \For{$j \gets low$ to $high - 1$}
                \If{$arr[j] \leq pivot$}
                    \State $i \gets i + 1$
                    \State \textbf{swap} $arr[i]$ \textbf{and} $arr[j]$
                \EndIf
            \EndFor
            \State \textbf{swap} $arr[i + 1]$ \textbf{and} $arr[high]$
            \State \Return $i + 1$
        \EndProcedure
    \end{algorithmic}
\end{algorithm}

\newpage
Now the exponent in the joint density in (4-11) can be simplified. By Result 4.9(a),

\begin{alignat*}{2}
    (\varx_j - \varmu)' \boldsigma^{-1} (\varx_j - \varmu) & = tr[(\varx_j - \varmu)' \boldsigma^{-1} (\varx_j - \varmu)] \\
    & = tr[\boldsigma^{-1} (\varx_j - \varmu) (\varx_j - \varmu)'] \tag*{(4-12)} \\
\end{alignat*}\\
Next, 
\begin{alignat*}{2}
    \sum_{j = 1}^{n} (\varx_j - \varmu)' \boldsigma^{-1} (\varx_j - \varmu) & = \sum_{j=1}^{n} tr[(\varx_j - \varmu)' \boldsigma^{-1} (\varx_j - \varmu)] \\
    & = \sum_{j=1}^{n} tr[\boldsigma^{-1} (\varx_j - \varmu) (\varx_j - \varmu)'] \\
    & = tr \left[ \boldsigma^{-1} \left( \sum_{j=1}^{n} (\varx_j - \varmu) (\varx_j - \varmu)' \right) \right] \tag{4-13} \label{eq:4-13} \\
\end{alignat*}

\end{document}