\documentclass[11pt]{article}
\usepackage[utf8]{vietnam}
\usepackage{mathtools}
\usepackage{amssymb}
\usepackage{amsthm}
\usepackage{algorithm}
\usepackage[noend]{algpseudocode}
\DeclareMathOperator{\Hamsin}{Sin}
\usepackage{listings}

\usepackage{xcolor}
\lstdefinestyle{mystyle}{
    backgroundcolor=\color{gray! 20},
    keywordstyle=\color{red},
    numbers=left,                    
    numbersep=5pt,
}
\lstset{style=mystyle}

\begin{document}
%bài trên lớp
\noindent
\textbf{Bài 1 :}\\
\begin{equation}
    x^2+y^2+z^2=0 \tag*{I}
\end{equation}
\textbf{Bài 2}
\begin{multline}
    F=\{ F_x \in F_c:(|S|>|C|) \cap\\ (MinPixels<|S|< maxPixels)\cap\\
    (|Sconnected|>|S|-\epsilon\}
\end{multline}
\textbf{Bài 3}
\begin{align*}
    &x=1 & x&y=2 & xyz=3 \\
    &xy=23 & x&yz=3 & x=11 \\
    &xyz=345 & &x=1 & xyz=22
\end{align*}
\textbf{Bài 4}
\begin{align}
    \boxed{x_1x_2x_3=\frac{-d}{a}} \tag*{[2]}\\
    x_1x_2+x_2x_3+x_3x_1= \frac{c}{a} \tag*{[3]}
    \intertext{Đồng thời}
    x_1 + x_2 + x_3 = \frac{-b}{a} 
    \tag*{[4]}
\end{align}
\textbf{Bài 5}

\begin{align}
    \overbrace{\overset{+}{A} + \underset{-}{B} + A^{B^{C^{2}}} + \frac{\omega}{\Phi}}^{\text{Vế trái}}= \underbrace{\underset{+}{A}+\underset{*}{B}+ 12\frac{A^B}{C}+ \frac{\Omega}{\varphi}}_{\text{Vế phải}} \tag{Pt.5} \label{Pt.5}
\end{align}
\newpage
\textbf{Bài 6}

\begin{equation}
\frac{\sqrt{3} + \sqrt{2}}{\sqrt{5} - \sqrt{7}} = 
\frac{
    \begin{split}
        (\sqrt{3} + \sqrt{2})(\sqrt{5} + \sqrt{7}) \\
        (\sqrt{5} + \sqrt{7})
    \end{split}
}{
    \begin{split}
        (\sqrt{5} - \sqrt{7})(\sqrt{5} + \sqrt{7}) \\
        (\sqrt{5} + \sqrt{7})
    \end{split}
}
\end{equation}

\textbf{Bài 7}
\lstdefinestyle{mystyle}{
    language=Python,
    backgroundcolor=\color{cyan!10!},
    keywordstyle=\color{blue},
    commentstyle=\color{green!60!black},
    stringstyle=\color{red},
    numberstyle=\tiny\color{gray},
    numbers=left,
    basicstyle=\ttfamily\small,
    showstringspaces=false,
    frame=single
}
\lstset{style = mystyle}
\begin{lstlisting}[caption={Mã nguồn Python}, label={lst:python-code}]
def factorial(n):
    if n == 0 or n == 1:
        return 1
    else:
        return n * factorial(n-1)

num = 5
result = factorial(num)
print(f"The factorial of {num} is {result}.")
\end{lstlisting}
\textbf{Bài 7}
\begin{algorithm}
\caption{QuickSort}
\label{alg:quicksort}
\begin{algorithmic}[1]
\Procedure{QuickSort}{$arr, low, high$}
    \If{$low < high$}
        \State $pivot \gets \Call{Partition}{arr, low, high}$
        \State $\Call{QuickSort}{arr, low, pivot - 1}$
        \State $\Call{QuickSort}{arr, pivot + 1, high}$
    \EndIf
\EndProcedure
\Statex
\Procedure{Partition}{$arr, low, high$}
    \State $pivot \gets arr[high]$
    \State $i \gets low - 1$
    \For{$j \gets low$ \textbf{to} $high - 1$}
        \If{$arr[j] \leq pivot$}
            \State $i \gets i + 1$
            \State \textbf{swap} $arr[i]$ \textbf{and} $arr[j]$
        \EndIf
    \EndFor
    \State \textbf{swap} $arr[i + 1]$ \textbf{and} $arr[high]$
    \State \textbf{return} $i + 1$
\EndProcedure
\end{algorithmic}
\end{algorithm}

\newpage
Now the exponent in the joint density in (4-11) can be simplified. By result 4.9(a),\\
\begin{align*}
    (x_j- \mu )'  \Sigma^{-1} (x_j-\mu)
    =&\text{tr}[(x_j-\mu)'\Sigma^{-1}(x_j-\mu)]\\
    =&\text{tr}[\Sigma^{-1}(x_j-\mu)(x_j-\mu)']
    \tag{4-12}
\end{align*}
Next,
\begin{align*}
     \sum_{j=1}^{n}(x_j-\mu)'\Sigma^{-1}(x_j-\mu)
 =&\sum_{j=1}^{n}\text{tr}[(x_j-\mu)'\Sigma^{-1}(x_j-\mu)]\\
 =&\sum_{j=1}^{n}\text{tr}[\Sigma^{-1}(x_j-\mu)(x_j-\mu)']\\
 =&\text{tr}\left[\Sigma^{-1}\left(\sum_{j=1}^{n}(x_j-\mu)(x_j-\mu)'\right)\right]
  \tag{4-13}
\end{align*}



\end{document}