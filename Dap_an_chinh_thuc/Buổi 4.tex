\documentclass[12pt]{article}
\usepackage[utf8]{vietnam}
\usepackage{array}
\usepackage{tabu}
\usepackage{multirow}
\usepackage{multicol}
\usepackage[table]{xcolor}
\usepackage{geometry}[left=1cm, right=1cm, top=3cm, bottom=3cm]
\setlength{\arrayrulewidth}{1pt}
\setlength{\tabcolsep}{2pt}
\renewcommand{\arraystretch}{3}

\begin{document}
%\textcolor{red}{2 trang đó}\\
\begin{table}[h]
    \centering
    \begin{tabular}{||b{0.3\linewidth}|b{0.3\linewidth}|b{0.3\linewidth}||}
    \hline
    {\bf Nhân tố sinh thái} &  {\bf Nhóm thực vật} & {\bf Nhóm động vật}\\[0.2cm]
    \hline
    \hline\vspace{0.2cm}
    {\bf Ánh sáng} & - Nhóm cây ưa sáng \newline - Nhóm cây ưa bóng & - Nhóm động vật ưa sáng \newline -Nhóm động vật ưa tối\\[0.2cm]
    \hline
    {\bf Nhiệt độ} & - Thực vật biến nhiệt \newline - Thực vật hằng nhiệt & - Động vật biến nhiệt \newline - Động vật hằng nhiệt\\[0.2cm]
    \hline
    {\bf Độ ẩm} & - Thực vật ưa ẩm \newline - Thực vật chịu hạn & - Động vật ưa ẩm \newline - Động vật ưa khô\\[0.2cm]
    \hline
    \end{tabular}
    \caption{Phân chia các nhóm sinh vật}
    \label{t1}
\end{table}
Bảng \ref{t1} cho ta thấy sự phân chia các nhóm sinh vật dựa vào các nhân tố sinh thái.


\arrayrulecolor{green!80!black!}
{\rowcolors{3}{green!50!yellow!50}{green!50!yellow!20}
\begin{tabular}{ |m{0.3\textwidth}|m{0.3\textwidth}|m{0.3\textwidth}|  }
\hline
\rowcolor{green!80!yellow!50}\multicolumn{3}{|c|}{Country List} \\
\hline
Country Name or Area Name& ISO ALPHA 2 Code &ISO ALPHA 3 \\
\hline
Afghanistan & AF &AFG \\
Aland Islands & AX   & ALA \\
Albania &AL & ALB \\
Algeria    &DZ & DZA \\
American Samoa & AS & ASM \\
Andorra & AD & AND   \\
Angola & AO & AGO \\
\hline
\end{tabular}}
\end{document}